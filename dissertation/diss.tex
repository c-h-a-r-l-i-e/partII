% Template for a Computer Science Tripos Part II project dissertation
\documentclass[12pt,a4paper,twoside,openright]{report}
\usepackage[pdfborder={0 0 0}]{hyperref}    % turns references into hyperlinks
\usepackage[margin=25mm]{geometry}  % adjusts page layout
\usepackage{graphicx}  % allows inclusion of PDF, PNG and JPG images
\usepackage{verbatim}
\usepackage{pdfpages}


\raggedbottom                           % try to avoid widows and orphans
\sloppy
\clubpenalty1000%
\widowpenalty1000%

\renewcommand{\baselinestretch}{1.1}    % adjust line spacing to make
                                        % more readable

\begin{document}

\bibliographystyle{plain}


%%%%%%%%%%%%%%%%%%%%%%%%%%%%%%%%%%%%%%%%%%%%%%%%%%%%%%%%%%%%%%%%%%%%%%%%
% Title


\pagestyle{empty}

\rightline{\LARGE \textbf{Charlie Maclean}}

\vspace*{60mm}
\begin{center}
\Huge
\textbf{Synthesis of Heart-Rate Detection Methods} \\[5mm]
Computer Science Tripos -- Part II \\[5mm]
King's College \\[5mm]
\today  % today's date
\end{center}

%%%%%%%%%%%%%%%%%%%%%%%%%%%%%%%%%%%%%%%%%%%%%%%%%%%%%%%%%%%%%%%%%%%%%%%%%%%%%%
% Proforma, table of contents and list of figures

\pagestyle{plain}

\chapter*{Proforma}

{\large
\begin{tabular}{ll}
Name:               & \bf Charlie Maclean                       \\
College:            & \bf King's College			\\
Project Title:      & \bf Synthesis of Heart-Rate Detection Methods	 \\
Examination:        & \bf Computer Science Tripos -- Part II, July 2020  \\
Word Count:         & \bf TODO\footnotemark[1]		\\
Project Originator: & Dr Robert Harle                   \\
Supervisor:         & Dr Robert Harle                   \\ 
\end{tabular}
}
\footnotetext[1]{This word count was computed
by \texttt{detex diss.tex | tr -cd '0-9A-Za-z $\tt\backslash$n' | wc -w}
}
\stepcounter{footnote}


\section*{Original Aims of the Project}

To research and implement the detection of heart rate from smartwatch 
sensors. To investigate the effectiveness of a selection of filters and 
peak finding algorithms. To use accelerometer data to find motion artifacts
within the data, and compare methods of removing these artifacts.


\section*{Work Completed}

All that has been completed appears in this dissertation.

\section*{Special Difficulties}

Learning how to incorporate encapulated postscript into a \LaTeX\
document on both Ubuntu Linux and OS X.
 
\newpage
\section*{Declaration}

I, Charlie Maclean of King's College, being a candidate for Part II of the 
Computer Science Tripos, hereby declare that this dissertation and the work 
described in it are my own work, unaided except as may be specified below, 
and that the dissertation does not contain material that has already been 
used to any substantial extent for a comparable purpose.

\bigskip
\leftline{Signed [signature]}

\medskip
\leftline{Date [date]}

\tableofcontents

\listoffigures

\newpage
\section*{Acknowledgements}

This document owes much to an earlier version written by Simon Moore
\cite{Moore95}.  His help, encouragement and advice was greatly 
appreciated.

%%%%%%%%%%%%%%%%%%%%%%%%%%%%%%%%%%%%%%%%%%%%%%%%%%%%%%%%%%%%%%%%%%%%%%%
% now for the chapters

\pagestyle{headings}

\chapter{Introduction}

Elite runners have historically used heart rate to provide an accurate 
measure of fitness, and allow them to train more effectively. [Expand on uses
of heart rate]. Previously, Electrocardiography (ECG) chest straps have been 
used to measure heart rate, by detecting the electrical signals controlling 
the expansion and contraction of the heart. They are accurate devices however
often prohibitively expensive, and hence inaccessible to casual runners.

In recent years, a new technology has emerged - Photoplethysmogram (PPG) 
- light is directed at the skin, and sensors measure how much blood vessels
scatter it. PPG sensors are cheaper than ECG sensors, and hence are 
available in a variety of products, particularly smartwatches. This 
innovation has bought a new wave of advanced training and monitoring onto
the wrists of any runner.

Switching from ECG to PPG is not without flaws though - ECG sensors return a
clean signal, as opposed to PPG signals which are contaminated with
noise. [More details about noise]. 

In particular when running, the motion can cause blood velocity to 
change, and the sensor can slip across the skin \cite{Wijshoff17}, 
\cite{Wood06}. These result in a distortion to the PPG signal, known as a
motion artifact (MA). Fortunately, smartwatches contain other sensors, such
as accelerometers and gyroscopes which can be used to predict the presence of 
MAs, and hence compensate for them.

My project is to research and develop a heart rate detection algorithm
for smartwatches worn during running. 

\chapter{Preparation}

In this chapter we introduce the relevant concepts for this project.


\section{Analysis of Existing Heart Rate Detection Methods}

?? Do we want this ??


\section{Gathering data}


\section{Filtering}

PPGs produce a large amount of noise from various sources:
\begin{itemize}
	\item One entry in the list
	\item Another entry in the list
\end{itemize}

In order to remove much of this noise, we can use filters to remove
frequencies we know are irrelevant. We know the heartbeat can vary from 
30 to 220 beats per minute, and hence we would like to disregard any noise 
outside of this range. In this section, I will introduce the concept of 
filters, and describe two filters useful for PPG signals.

\begin{figure}[h!]
	\centerline{\includegraphics[width=0.8\textwidth]{figs/filter.png}}
\caption{Diagram showing filter characteristics}
\label{fig:filterdiag}
\end{figure}

Figure ~\ref{fig:filterdiag} displays an example filter magnitude response
diagram. The characteristics displayed are as follows.

The passband is the range of frequencies we would like to remain. In an ideal
filter, there is no loss within the passband.

Passband ripple describes the variation in amplitude \emph{within} the
passband. An ideal filter will have no passband filter, such that all
frequencies within the passband are permitted equally.

The transition width is the frequency range between the start and stop band.
Ideally, this is zero, such that frequencies outside of the passband are
instantly reduced.

The stopband is the range of frequencies we wish to remove. An ideal filter
completely removes all frequencies within the stopband.

The ideal filter, as described above, is impossible - and hence we have a
variety of filters which compromise between the desirable characteristics.
Following this, I detail two of these compromises - the Butterworth filter,
and the Chebyshev filter.


\subsection{Butterworth Filter}

The Butterworth filter aims to minimize passband ripple, at the expense of a
larger transition width. 

\subsection{Chebyshev Filter}

Chebyshev designed a filter which aimed to provide minimum transition width,
at the cost of a larger passband ripple.

\section{Peak detection}

Once we have a clean signal, we need to figure out the points in the signal
that correspond to a heart beat. In PGG, we get a signal with multiple
features.
%TODO: give information about these features.

There are multiple approaches to finding peaks:

\subsection{Local maxima}
\subsection{Wavelet transformation}


\section{Motion artifact reduction}

In this section I explore methods that can be used to reduce the effect of 
Motion Artifacts (MAs) on the PPG Signal.

\subsection{Adaptive Noise Cancellation}

Adaptive Noise Cancellation (ANC), as described by Widrow et al.
\cite{Widrow75}, is a technique which allows us to remove noise from a signal,
given we have another signal correlated to the noise in some way. In our 
situation, we know MAs are correlated to motion, so we can use the 
accelerometer to remove MAs. This technique is extremely useful, as it can 
still produce good results when the MAs are at the same frequency as the heart
beat, even if the MAs have a larger amplitude.


\begin{figure}[tbh]
	\centerline{\includegraphics[width=\textwidth]{figs/ANC-concept.png}}
\caption{ANC overall flow}
\label{epsfig}
\end{figure}

An overview of the algorithm follows. We have signal from heartbeat \(s\) that
we want to figure out, but this is contaminated by noise from MAs \(n_0\).
We assume the PPG sensor reading \(p\) is such that \(p=s+n_0\). Additionally,
we have accelerometer sensor readings \(n_1\). We filter \(n_1\), producing
signal \(y\) and subtract this from \(p\), to produce \(z=p-y\). Our aim is to
adjust the filter, such that \(y\) is as close to \(n_0\) as possible. 

The adaptive filter aims to choose a filter such that the power output 
\(E[z^2]\) is minimized. Given \(s\) is uncorrelated with \(n_0\) and \(n_1\), and \(n_0\)
is correlated with \(n_1\), it can be proven that minimizing the output
power is equivilant to 
% TODO: do this maths better

\chapter{Implementation}

\section{Verbatim text}

Verbatim text can be included using \verb|\begin{verbatim}| and
\verb|\end{verbatim}|. I normally use a slightly smaller font and
often squeeze the lines a little closer together, as in:

{\renewcommand{\baselinestretch}{0.8}\small
\begin{verbatim}
GET "libhdr"
 
GLOBAL { count:200; all  }
 
LET try(ld, row, rd) BE TEST row=all
                        THEN count := count + 1
                        ELSE { LET poss = all & ~(ld | row | rd)
                               UNTIL poss=0 DO
                               { LET p = poss & -poss
                                 poss := poss - p
                                 try(ld+p << 1, row+p, rd+p >> 1)
                               }
                             }
LET start() = VALOF
{ all := 1
  FOR i = 1 TO 12 DO
  { count := 0
    try(0, 0, 0)
    writef("Number of solutions to %i2-queens is %i5*n", i, count)
    all := 2*all + 1
  }
  RESULTIS 0
}
\end{verbatim}
}

\section{Tables}

\begin{samepage}
Here is a simple example\footnote{A footnote} of a table.

\begin{center}
\begin{tabular}{l|c|r}
Left      & Centred & Right \\
Justified &         & Justified \\[3mm]
%\hline\\%[-2mm]
First     & A       & XXX \\
Second    & AA      & XX  \\
Last      & AAA     & X   \\
\end{tabular}
\end{center}

\noindent
There is another example table in the proforma.
\end{samepage}

\section{Simple diagrams}

Simple diagrams can be written directly in \LaTeX.  For example, see
figure~\ref{latexpic1} on page~\pageref{latexpic1} and see
figure~\ref{latexpic2} on page~\pageref{latexpic2}.

\begin{figure}
\setlength{\unitlength}{1mm}
\begin{center}
\begin{picture}(125,100)
\put(0,80){\framebox(50,10){AAA}}
\put(0,60){\framebox(50,10){BBB}}
\put(0,40){\framebox(50,10){CCC}}
\put(0,20){\framebox(50,10){DDD}}
\put(0,00){\framebox(50,10){EEE}}

\put(75,80){\framebox(50,10){XXX}}
\put(75,60){\framebox(50,10){YYY}}
\put(75,40){\framebox(50,10){ZZZ}}

\put(25,80){\vector(0,-1){10}}
\put(25,60){\vector(0,-1){10}}
\put(25,50){\vector(0,1){10}}
\put(25,40){\vector(0,-1){10}}
\put(25,20){\vector(0,-1){10}}

\put(100,80){\vector(0,-1){10}}
\put(100,70){\vector(0,1){10}}
\put(100,60){\vector(0,-1){10}}
\put(100,50){\vector(0,1){10}}

\put(50,65){\vector(1,0){25}}
\put(75,65){\vector(-1,0){25}}
\end{picture}
\end{center}
\caption{A picture composed of boxes and vectors.}
\label{latexpic1}
\end{figure}

\begin{figure}
\setlength{\unitlength}{1mm}
\begin{center}

\begin{picture}(100,70)
\put(47,65){\circle{10}}
\put(45,64){abc}

\put(37,45){\circle{10}}
\put(37,51){\line(1,1){7}}
\put(35,44){def}

\put(57,25){\circle{10}}
\put(57,31){\line(-1,3){9}}
\put(57,31){\line(-3,2){15}}
\put(55,24){ghi}

\put(32,0){\framebox(10,10){A}}
\put(52,0){\framebox(10,10){B}}
\put(37,12){\line(0,1){26}}
\put(37,12){\line(2,1){15}}
\put(57,12){\line(0,2){6}}
\end{picture}

\end{center}
\caption{A diagram composed of circles, lines and boxes.}
\label{latexpic2}
\end{figure}



\section{Adding more complicated graphics}

The use of \LaTeX\ format can be tedious and it is often better to use
encapsulated postscript (EPS) or PDF to represent complicated graphics.
Figure~\ref{epsfig} and~\ref{xfig} on page \pageref{xfig} are
examples. The second figure was drawn using \texttt{xfig} and exported in
{\tt.eps} format. This is my recommended way of drawing all diagrams.


\begin{figure}[tbh]
\centerline{\includegraphics{figs/cuarms.pdf}}
\caption{Example figure using encapsulated postscript}
\label{epsfig}
\end{figure}

\begin{figure}[tbh]
\vspace{4in}
\caption{Example figure where a picture can be pasted in}
\label{pastedfig}
\end{figure}


\begin{figure}[tbh]
\centerline{\includegraphics{figs/diagram.pdf}}
\caption{Example diagram drawn using \texttt{xfig}}
\label{xfig}
\end{figure}


\chapter{Evaluation}

\section{Printing and binding}

Use a ``duplex'' laser printer that can print on both sides to print
two copies of your dissertation. Then bind them, for example using the
comb binder in the Computer Laboratory Library.

\section{Further information}

See the Unix Tools notes at

\url{http://www.cl.cam.ac.uk/teaching/current-1/UnixTools/materials.html}


\chapter{Conclusion}

I hope that this rough guide to writing a dissertation is \LaTeX\ has
been helpful and saved you time.


%%%%%%%%%%%%%%%%%%%%%%%%%%%%%%%%%%%%%%%%%%%%%%%%%%%%%%%%%%%%%%%%%%%%%
% the bibliography
\addcontentsline{toc}{chapter}{Bibliography}
\bibliography{refs}

%%%%%%%%%%%%%%%%%%%%%%%%%%%%%%%%%%%%%%%%%%%%%%%%%%%%%%%%%%%%%%%%%%%%%
% the appendices
\appendix


\chapter{Latex source}

\section{diss.tex}
{\scriptsize\verbatiminput{diss.tex}}

\chapter{Makefile}

\section{makefile}\label{makefile}
{\scriptsize\verbatiminput{makefile.txt}}

\section{refs.bib}
{\scriptsize\verbatiminput{refs.bib}}


\chapter{Project Proposal}

\includepdf[pages=-,pagecommand={},width=\textwidth]{proposal.pdf}

\end{document}
